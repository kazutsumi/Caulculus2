\documentclass[10pt, uplatex, dvipdfmx]{jsarticle}
\usepackage{../mypackage} 

\graphicspath{{../pictures}} 

\setcounter{section}{10}

\begin{document}

\section{Gauss 積分と誤差関数}

\subsection{Gauss 積分}

以下の広義積分は\textbf{Gauss 積分}と呼ばれる.この値を広義2重積分を使って求める.
\[
  \int_{-\infty}^{\infty} e^{-x^2} \ dx
\]

\vspace{1zh}

まず,この広義積分 $I$ が収束することを確認しておく.開区間上の広義積分の定義に従って
\begin{equation}\label{eq:gauss-divide}
  \int_{-\infty}^{\infty} e^{-x^2} \ dx  = \int_{-\infty}^{0} e^{-x^2} \ dx + \int_{0}^{\infty} e^{-x^2} \ dx
\end{equation}
と分けて,右辺の2個の広義積分がいずれも収束することを示す.ここで,$u=-x$ としての置換積分により
\begin{equation}\label{eq:gauss-L}
  \int_{-\infty}^{0} e^{-x^2} \ dx = \lim_{c \to -\infty} \int_{c}^{0} e^{-x^2} \ dx
  = \lim_{c \to -\infty} \int_{-c}^{0} -e^{-u^2} \ du = \lim_{c \to \infty} \int_{0}^{-c} e^{-x^2} \ dx
  =\int_{0}^{\infty} e^{-x^2} \ dx
\end{equation}
なので,(\ref{eq:gauss-divide})の右辺第2項が収束することを示せばよい.さらに,
\begin{equation}\label{eq:gauss-R}
  \int_{0}^{\infty} e^{-x^2} \ dx = \int_{0}^{1} e^{-x^2} \ dx + \int_{1}^{\infty} e^{-x^2} \ dx
\end{equation}
と分ければ,この右辺第1項は通常の積分なので第2項が収束することを示せばよい.$x \in [1, \infty)$ に対して
\[
  \left| e^{-x^2} \right| = e^{-x^2} \leqq x e^{-x^2}
\]
である.さらに,この最右辺の $[1,\infty)$ 上の広義積分は以下のように収束する.
\[
  \int_{1}^{\infty} x e^{-x^2} \ dx = \lim_{c \to \infty} \int_{1}^{c}
  x e^{-x^2} \ dx = \lim_{c \to \infty} \left[
    -\frac{e^{-x^2}}{2}\right]_{1}^{c} = \lim_{c \to \infty}
  \frac{e^{-1} - e^{-c^2}}{2} = \frac{e^{-1}}{2}
\]
よって,(\ref{eq:gauss-R})の右辺第2項の
広義積分は収束するので,Gauss 積分は収束する.\\

上記の通り Gauss 積分は収束するので,以下が成り立つ.
\[
  \int_{-\infty}^{\infty} e^{-x^2} \ dx  = \lim_{n \to \infty} \int_{-n}^{n} e^{-x^2} \ dx
\]
なお,一般には
\[
  \int_{-\infty}^{\infty} f(x) \ dx \text{ と } \lim_{n \to \infty} \int_{-n}^{n} f(x) \ dx
\]
は別物だが,広義積分 $\ds \int_{-\infty}^{\infty} f(x) \ dx$ が収束する
なら,両者は等しい.従って,以下が成り立つ.
\begin{equation}\label{eq:gauss-square}
  \begin{aligned}
    \left( \int_{-\infty}^{\infty} e^{-x^2} \ dx \right)^2
    &= \left( \lim_{n \to \infty} \int_{-n}^{n} e^{-x^2} \ dx \right)^2
      = \lim_{n \to \infty} \left( \int_{-n}^{n} e^{-x^2} \ dx\right)^2\\[1ex]
    & = \lim_{n \to \infty} \left( \int_{-n}^{n} e^{-x^2} \ dx \right) \left( \int_{-n}^{n} e^{-y^2} \ dy \right)
      = \lim_{n \to \infty} \int_{-n}^{n} \left( \int_{-n}^{n} e^{-x^2-y^2} \ dx \right) dy\\[1ex]
    &= \lim_{n \to \infty} \iint_{[-n,n] \times [-n,n]} e^{-x^2-y^2} \ dx dy
  \end{aligned}
\end{equation}
ここで,$D_n = [-n,n] \times [-n,n]$ とおけ
ば,$\Set{D_n}$ は $\mathbb{R}^2$ の増加近似列である.任意の $(x,y)
\in \mathbb{R}^2$ に対して
$e^{-x^2-y^2}>0$ なので,(\ref{eq:gauss-square})から以下が成り立つ.
\begin{equation}
  \left( \int_{-\infty}^{\infty} e^{-x^2} \ dx \right)^2 = \iint_{\mathbb{R}^2} e^{-x^2-y^2} \ dx dy
\end{equation}
この右辺の広義2重積分の値を求める.これが収束することは既にわかっている
ので,$\mathbb{R}^2$ の増加近似列として計算しやすいものを選べばよい.
\[
  E_n = \Set{(x,y) \ \mid \ x^2+y^2 \leqq n^2}
\]
とすれば,$\Set{E_n}$ は $\mathbb{R}^2$ の増加近似列なので
\[
  \iint_{\mathbb{R}^2} e^{-x^2-y^2} \ dx dy = \lim_{n \to \infty} \iint_{E_n} e^{-x^2-y^2} \ dx dy
\]
である.極座標変換 $x=r\cos\theta, \; y=r\sin\theta$ によって,$r\theta$ 平面の閉領域
\[
  F_n = \Set{(r, \theta) \ \mid \ 0 \leqq r \leqq n, \; 0 \leqq \theta \leqq 2\pi}
\]
が $xy$ 平面の $E_n$ に変換される.極座標変換の Jacobian は $J(r,\theta)=r$ なので,
\[
  \begin{aligned}
    \iint_{E_n} e^{-x^2-y^2} \ dx dy
    &= \iint_{F_n} e^{-r^2} |J(r,\theta)| \ dr d\theta
      = \int_{0}^{2\pi} \left( \int_{0}^{n} r e^{-r^2} \ dr \right) d\theta\\[1ex]
    & = \left( \int_{0}^{2\pi} d\theta\right) \left( \int_{0}^{n} r e^{-r^2} \ dr\right)
      = 2\pi \left[ -\frac{e^{-r^2}}{2}\right]_{0}^{n} = \pi \left( 1 - e^{-n^2}\right)
  \end{aligned}
\]
である.よって,以下を得る.
\[
  \iint_{\mathbb{R}^2} e^{-x^2-y^2} \ dx dy = \lim_{n \to \infty} \pi \left( 1-e^{-n^2}\right) = \pi
\]
これと(\ref{eq:gauss-square})から
$\ds \left( \int_{-\infty}^{\infty} e^{-x^2} \ dx \right)^2 = \pi$ であ
り,明らかに $\ds \int_{-\infty}^{\infty} e^{-x^2} \ dx >0$ なので,Gauss積分の値は
\begin{equation}\label{eq:gauss-int}
  \fbox{$\ds \int_{-\infty}^{\infty} e^{-x^2} \ dx = \sqrt{\pi}$}
\end{equation}
である.さらに,(\ref{eq:gauss-divide})と(\ref{eq:gauss-L})から
\[
  \int_{-\infty}^{\infty} e^{-x^2} \ dx = 2 \int_{0}^{\infty} e^{-x^2} \ dx
\]
なので,(\ref{eq:gauss-int})と合わせて以下も得られる.
\begin{equation}\label{eq:gauss-half}
  \fbox{$\ds \int_{0}^{\infty} e^{-x^2} \ dx = \frac{\sqrt{\pi}}{2}$}
\end{equation}

次に紹介する誤差関数$\ds \erf x = \frac{2}{\sqrt{\pi}}\int_{0}^{x}
e^{-t^2} \ dt$ に$\ds \frac{2}{\sqrt{\pi}}$ という定数が含まれているの
はこの(\ref{eq:gauss-half})が理由である.つまり,$\ds \lim_{x \to
  \infty} \erf x = 1$ となって欲しいからである.

\newpage

\subsection{誤差関数}

$\mathbb{R}$ 上定義された次の関数は\textbf{誤差関数(error function)}と
呼ばれる.
\[
  \erf x = \frac{2}{\sqrt{\pi}} \int_{0}^{x} e^{-t^2} \ dt
\]
定義とガウス積分に関する結果 (\ref{eq:gauss-half}) から以下が直ちにわかる.
\begin{equation}\label{eq:prop-erf}
   \erf 0 = 0, \quad \erf(-x) = -\erf(x), \quad \lim_{x \to \infty} \erf x = 1, \quad \lim_{x \to -\infty} \erf x =-1
\end{equation}
また,微分積分学の基本定理から
\[
  \frac{d}{dx} \erf x = \frac{2}{\sqrt{\pi}} e^{-x^2}
\]
である.つまり,以下の積分の公式が得られる.なお,積分定数は省略している.
\begin{equation}\label{eq:int-exp-x2}
  \int e^{-x^2} \ dx = \frac{\sqrt{\pi}}{2} \erf x
\end{equation}
初等関数の組み合わせでは表せない $e^{-x^2}$ の原始関数が誤差関
数を用いて表せているが,実際には $e^{-x^2}$ の原始関数(の定数倍)に誤差
関数という名前を付けただけである.

\begin{example}\label{exmp:x2exp}
  これまで扱いづらかった不定積分を $\erf$ を使って表すこともできる.
  \[
    \int x^2 e^{-x^2} \ dx = \int x \cdot xe^{-x^2}\ dx 
    = x \left( -\frac{1}{2} e^{-x^2}\right) + \frac{1}{2} \int e^{-x^2} \ dx
    = -\frac{1}{2} x e^{-x^2} + \frac{\sqrt{\pi}}{4} \erf x
  \]
\end{example}

\vspace{1zh}


\subsection{正規分布の累積分布関数と誤差関数}

正規分布 $N(\mu, \sigma^2)$ の確率密度関数 $f_{\mu, \sigma^2}$ は次の式
で与えられる.なお,$\exp (X) = e^X$ である.
\[
  f_{\mu, \sigma^2} (x) = \frac{1}{\sqrt{2\pi\sigma^2}}\exp\left(-\frac{(x-\mu)^2}{2\sigma^2}\right)
\]
正規分布 $N(\mu, \sigma^2)$ に従う確率変数 $X$ が $a \leqq X \leqq b$
である確率 $P_{\mu, \sigma^2}(a\leqq X \leqq b)$ は
\[
  P_{\mu,\sigma^2}(a \leqq X \leqq b) = \int_{a}^{b} f_{\mu, \sigma^2}(x) \ dx  
  = \frac{1}{\sqrt{2\pi \sigma^2}}\int_{a}^{b} \exp\left(-\frac{(x-\mu)^2}{2\sigma^2}\right)\ dx
\]
である.特に,正規分布の累積分布関数,つまり $X \leqq x$ である確
率 $P_{\mu, \sigma^2}(X \leqq x)$ は広義積分として
\[
  P_{\mu, \sigma^2}(X \leqq x) = \int_{-\infty}^{x} f_{\mu, \sigma^2}(t) \ dt
  = \frac{1}{\sqrt{2\pi \sigma^2}} \int_{-\infty}^{x}
  \exp\left( -\frac{(t-\mu)^2}{2\sigma^2}\right) \ dt
\]
と書ける.これを誤差関数 $\erf$ で表してみる.変数変換
\[
  u=\frac{t-\mu}{\sqrt{2\sigma^2}}\;  \left(\Leftrightarrow t= \mu + \sqrt{2\sigma^2} \ u\right)
\]
による置換積分によって,$P_{\mu,\sigma^2}(X \leqq x)$ は次のように書き換えられる.
\begin{equation}\label{eq:P-erf}
  P_{\mu,\sigma^2} (X \leq x)= \frac{1}{\sqrt{\pi}} \int_{-\infty}^{\frac{x-\mu}{\sqrt{2\sigma^2}}} e^{-u^2}\ du
  =\frac{1}{2}\Big[\erf u\Big]_{-\infty}^{\frac{x-\mu}{\sqrt{2\sigma^2}}} 
  = \frac{1}{2} + \frac{1}{2}\erf \frac{x-\mu}{\sqrt{2\sigma^2}}
\end{equation}
特に $\mu=0, \; \sigma=1$ のとき,つまり標準正規分布 $N(0,1)$ の累積分布関数は次のように書ける.
\[
  P_{0,1}(X \leqq x) = \frac{1}{2} + \frac{1}{2} \erf \frac{x}{\sqrt{2}}
\]
確率・統計の書籍では標準正規分布表として,この $P_{0,1}(X \leqq x)$ の
値,あるいは $P_{0,1}(0 \leqq X \leqq x)$ の値の表が巻末に載っているこ
とが多い.ここで,
\[
  P_{0,1}( x_1 \leqq X \leqq x_2) = P_{0,1}(X \leqq x_2) - P_{0,1}(X \leqq x_1), \quad P_{0,1}(X\leqq 0) = \frac{1}{2}
\]
より,$x \geqq 0$ のときはこれらの間に
\[
  P_{0,1}(X \leqq x) = P_{0,1}( 0 \leqq X \leqq x) + \frac{1}{2}
\]
という関係があるので,$P_{0,1}(X \leqq x)$ と $P_{0,1}(0 \leqq X \leqq x)$ の一方から
他方の値は分かる.さらに,(\ref{eq:P-erf}) から
\[
  \erf x = 2P_{0,1}\left(X \leqq \sqrt{2}x\right) - 1 = 2P_{0,1}\left( 0 \leqq X \leqq \sqrt{2}x\right)
\]
なので,$\erf$ の値は標準正規分布表から読み取れるし,逆に $\erf$ から標準正規分布表を自作することもできる.



\begin{example}[\textbf{正規分布の平均}]
正規分布 $N(\mu, \sigma^2)$ の平均(期待値)は以下の広義積分で定義される.
\[
  \int_{-\infty}^{\infty} x f_{\mu, \sigma^2}(x) \ dx 
  = \frac{1}{\sqrt{2\pi\sigma^2}}\int_{-\infty}^{\infty}x \exp\left(-\frac{(x-\mu)^2}{2\sigma^2}\right) \ dx
\]
これが $\mu$ に等しいことは,変数変
換
\begin{equation}
  \label{eq:x-mu-sigma-u}
  u = \frac{x-\mu}{\sqrt{2\sigma^2}} \; \left( \Leftrightarrow x = \mu+\sqrt{2\sigma^2}~u\right)
\end{equation}
での置換積分とガウス積分 (\ref{eq:gauss-int})から以下のように確かめられる.
\[
  \int_{-\infty}^{\infty} x f_{\mu, \sigma^2}(x)\ dx 
  = \frac{1}{\sqrt{\pi}}\int_{-\infty}^{\infty} \left( \mu + \sqrt{2\sigma^2}\ u\right) e^{-u^2} du
  = \frac{\mu}{\sqrt{\pi}}\int_{-\infty}^{\infty} e^{-u^2} du 
  + \sqrt{\frac{2\sigma^2}{\pi}} \underset{=0}{\uwave{\int_{-\infty}^{\infty} u e^{-u^2} du}}=\mu
\]
\end{example}

\begin{example}[\textbf{正規分布の分散}]
正規分布 $N(\mu, \sigma^2)$ の分散は以下の広義積分で定義される.
\[
  \int_{-\infty}^{\infty} (x-\mu)^2 f_{\mu,\sigma^2}(x)\ dx 
  = \frac{1}{\sqrt{2\pi\sigma^2}} \int_{-\infty}^{\infty} (x-\mu)^2 \exp\left(-\frac{(x-\mu)^2}{2\sigma^2}\right)\ dx
\]
これが $\sigma^2$ に等しいことは,変数変換 (\ref{eq:x-mu-sigma-u})での
置換積分と例\ref{exmp:x2exp}の結果および (\ref{eq:prop-erf}) から以下のように確かめられる.
\[
  \begin{aligned}
    \int_{-\infty}^{\infty}(x-\mu)^2f_{\mu,\sigma^2}(x) dx 
    &= \frac{2\sigma^2}{\sqrt{\pi}}\int_{-\infty}^{\infty} u^2 e^{-u^2}\ du
      =\frac{4\sigma^2}{\sqrt{\pi}} \int_{0}^{\infty} u^2 e^{-u^2}\ du\\[1ex]
    &= \frac{4\sigma^2}{\sqrt{\pi}} \lim_{c \to \infty} \left[ - \frac{1}{2}u e^{-u^2}
      + \frac{\sqrt{\pi}}{4} \erf u \right]_{0}^{c}
      = \sigma^2 \lim_{c \to \infty} \erf c
      = \sigma^2
  \end{aligned}
\]
\end{example}

\end{document}
