\documentclass[10pt, uplatex, dvipdfmx]{jsarticle}
\usepackage{../mypackage} 

\graphicspath{{../pictures}} 

\setcounter{section}{10}

\begin{document}

\section{Gauss 積分と誤差関数}

\subsection{Gauss 積分}

以下の広義積分は\textbf{Gauss 積分}と呼ばれる.この値を広義2重積分を使って求める.
\[
  \int_{-\infty}^{\infty} e^{-x^2} \ dx
\]

\vspace{1zh}

まず,この広義積分 $I$ が収束することを確認しておく.開区間上の広義積分の定義に従って
\begin{equation}\label{eq:gauss-divide}
  \int_{-\infty}^{\infty} e^{-x^2} \ dx  = \int_{-\infty}^{0} e^{-x^2} \ dx + \int_{0}^{\infty} e^{-x^2} \ dx
\end{equation}
と分けて,右辺の2個の広義積分がいずれも収束することを示す.ここで,$u=-x$ としての置換積分により
\begin{equation}\label{eq:gauss-L}
  \int_{-\infty}^{0} e^{-x^2} \ dx = \lim_{c \to -\infty} \int_{c}^{0} e^{-x^2} \ dx
  = \lim_{c \to -\infty} \int_{-c}^{0} -e^{-u^2} \ du = \lim_{c \to \infty} \int_{0}^{-c} e^{-x^2} \ dx
  =\int_{0}^{\infty} e^{-x^2} \ dx
\end{equation}
なので,(\ref{eq:gauss-divide})の右辺第2項が収束することを示せばよい.さらに,
\begin{equation}\label{eq:gauss-R}
  \int_{0}^{\infty} e^{-x^2} \ dx = \int_{0}^{1} e^{-x^2} \ dx + \int_{1}^{\infty} e^{-x^2} \ dx
\end{equation}
と分ければ,この右辺第1項は通常の積分なので第2項が収束することを示せばよい.$x \in [1, \infty)$ に対して
\[
  \left| e^{-x^2} \right| = e^{-x^2} \leqq x e^{-x^2}
\]
である.さらに,この最右辺の $[1,\infty)$ 上の広義積分は以下のように収束する.
\[
  \int_{1}^{\infty} x e^{-x^2} \ dx = \lim_{c \to \infty} \int_{1}^{c}
  x e^{-x^2} \ dx = \lim_{c \to \infty} \left[
    -\frac{e^{-x^2}}{2}\right]_{1}^{c} = \lim_{c \to \infty}
  \frac{e^{-1} - e^{-c^2}}{2} = \frac{e^{-1}}{2}
\]
よって,(\ref{eq:gauss-R})の右辺第2項の
広義積分は収束するので,Gauss 積分は収束する.\\

上記の通り Gauss 積分は収束するので,以下が成り立つ.
\[
  \int_{-\infty}^{\infty} e^{-x^2} \ dx  = \lim_{n \to \infty} \int_{-n}^{n} e^{-x^2} \ dx
\]
なお,一般には
\[
  \int_{-\infty}^{\infty} f(x) \ dx \text{ と } \lim_{n \to \infty} \int_{-n}^{n} f(x) \ dx
\]
は別物だが,広義積分 $\ds \int_{-\infty}^{\infty} f(x) \ dx$ が収束する
なら,両者は等しい.従って,以下が成り立つ.
\begin{equation}\label{eq:gauss-square}
  \begin{aligned}
    \left( \int_{-\infty}^{\infty} e^{-x^2} \ dx \right)^2
    &= \left( \lim_{n \to \infty} \int_{-n}^{n} e^{-x^2} \ dx \right)^2
      = \lim_{n \to \infty} \left( \int_{-n}^{n} e^{-x^2} \ dx\right)^2\\[1ex]
    & = \lim_{n \to \infty} \left( \int_{-n}^{n} e^{-x^2} \ dx \right) \left( \int_{-n}^{n} e^{-y^2} \ dy \right)
      = \lim_{n \to \infty} \int_{-n}^{n} \left( \int_{-n}^{n} e^{-x^2-y^2} \ dx \right) dy\\[1ex]
    &= \lim_{n \to \infty} \iint_{[-n,n] \times [-n,n]} e^{-x^2-y^2} \ dx dy
  \end{aligned}
\end{equation}
ここで,$D_n = [-n,n] \times [-n,n]$ とおけ
ば,$\Set{D_n}$ は $\mathbb{R}^2$ の増加近似列である.任意の $(x,y)
\in \mathbb{R}^2$ に対して
$e^{-x^2-y^2}>0$ なので,(\ref{eq:gauss-square})から以下が成り立つ.
\begin{equation}
  \left( \int_{-\infty}^{\infty} e^{-x^2} \ dx \right)^2 = \iint_{\mathbb{R}^2} e^{-x^2-y^2} \ dx dy
\end{equation}
この右辺の広義2重積分の値を求める.これが収束することは既にわかっている
ので,$\mathbb{R}^2$ の増加近似列として計算しやすいものを選べばよい.
\[
  E_n = \Set{(x,y) \ \mid \ x^2+y^2 \leqq n^2}
\]
とすれば,$\Set{E_n}$ は $\mathbb{R}^2$ の増加近似列なので
\[
  \iint_{\mathbb{R}^2} e^{-x^2-y^2} \ dx dy = \lim_{n \to \infty} \iint_{E_n} e^{-x^2-y^2} \ dx dy
\]
である.極座標変換 $x=r\cos\theta, \; y=r\sin\theta$ によって,$r\theta$ 平面の閉領域
\[
  F_n = \Set{(r, \theta) \ \mid \ 0 \leqq r \leqq n, \; 0 \leqq \theta \leqq 2\pi}
\]
が $xy$ 平面の $E_n$ に変換される.極座標変換の Jacobian は $J(r,\theta)=r$ なので,
\[
  \begin{aligned}
    \iint_{E_n} e^{-x^2-y^2} \ dx dy
    &= \iint_{F_n} e^{-r^2} |J(r,\theta)| \ dr d\theta
      = \int_{0}^{2\pi} \left( \int_{0}^{n} r e^{-r^2} \ dr \right) d\theta\\[1ex]
    & = \left( \int_{0}^{2\pi} d\theta\right) \left( \int_{0}^{n} r e^{-r^2} \ dr\right)
      = 2\pi \left[ -\frac{e^{-r^2}}{2}\right]_{0}^{n} = \pi \left( 1 - e^{-n^2}\right)
  \end{aligned}
\]
である.よって,以下を得る.
\[
  \iint_{\mathbb{R}^2} e^{-x^2-y^2} \ dx dy = \lim_{n \to \infty} \pi \left( 1-e^{-n^2}\right) = \pi
\]
これと(\ref{eq:gauss-square})から
$\ds \left( \int_{-\infty}^{\infty} e^{-x^2} \ dx \right)^2 = \pi$ であ
り,明らかに $\ds \int_{-\infty}^{\infty} e^{-x^2} \ dx >0$ なので,Gauss積分の値は
\begin{equation}\label{eq:gauss-int}
  \fbox{$\ds \int_{-\infty}^{\infty} e^{-x^2} \ dx = \sqrt{\pi}$}
\end{equation}
である.さらに,(\ref{eq:gauss-divide})と(\ref{eq:gauss-L})から
\[
  \int_{-\infty}^{\infty} e^{-x^2} \ dx = 2 \int_{0}^{\infty} e^{-x^2} \ dx
\]
なので,(\ref{eq:gauss-int})と合わせて以下も得られる.
\begin{equation}\label{eq:gauss-half}
  \fbox{$\ds \int_{0}^{\infty} e^{-x^2} \ dx = \frac{\sqrt{\pi}}{2}$}
\end{equation}
次に紹介する誤差関数$\ds \erf x = \frac{2}{\sqrt{\pi}}\int_{0}^{x}
e^{-t^2} \ dt$ に$\ds \frac{2}{\sqrt{\pi}}$ という定数が含まれているの
はこの(\ref{eq:gauss-half})が理由である.つまり,$\ds \lim_{x \to
  \infty} \erf x = 1$ となって欲しいからである.

\newpage

\subsection{誤差関数}

\end{document}
